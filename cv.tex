  % !TEX program = xelatex
  \documentclass[a4paper,10pt]{article}

  \usepackage[icelandic,english]{babel}
  % SG only icelandic (for now)
  \selectlanguage{icelandic}

  \usepackage{etoolbox}

  % Essential packages
  \usepackage{fontspec}
  \usepackage{xunicode}
  \usepackage{xcolor}
  \usepackage{hyperref}
  \usepackage{qrcode}
  \usepackage{enumitem}
  \usepackage{fontawesome5}
  \usepackage{geometry}
  \usepackage{titlesec}
  \usepackage{array}
  \usepackage{multicol}
  \usepackage{needspace}
  \usepackage{svg}
  % \usepackage{tikz}
  \usepackage{soul} % For underline customization
  \usepackage{xurl}  % For better URL breaking
  \usepackage{fvextra}  % For better monospace handling

 
  % Translation system
  \newcommand{\tr}[2]{%
    \iflanguage{icelandic}{#1}{#1}%
  }

  \newcommand{\sectionMe}{\tr{Ég}{Me}}
  \newcommand{\sectionExperience}{\tr{Reynsla}{Professional Experience}}
  \newcommand{\sectionSkills}{\tr{Hæfni}{Technical Skills}}
  \newcommand{\sectionProjects}{\tr{Verkefni}{Notable Projects}}
  \newcommand{\sectionCertifications}{\tr{Námskeið og Vottanir}{Certifications}}
  \newcommand{\sectionAwards}{\tr{Viðurkenningar}{Awards}}
  \newcommand{\sectionReferences}{\tr{Meðmælendur}{References}}


  % In the preamble, define a consistent spacing
  \newlength{\qrspacing}
  \setlength{\qrspacing}{1em}  % Adjust this value as needed


  % Page geometry - compact margins
  \geometry{
    top=1.2cm,
    bottom=1.2cm,
    left=1.5cm,
    right=1.5cm
  }

  % Colors
  \definecolor{primarytext}{HTML}{000000}
  \definecolor{secondarytext}{HTML}{4B5563}
  \definecolor{c02529c}{RGB}{2,82,156}
  \definecolor{cdc1e35}{RGB}{220,30,53}
  \definecolor{c003897}{RGB}{0,56,151}
  \definecolor{cd72828}{RGB}{215,40,40}
  \definecolor{linkcolor}{HTML}{4A5568}  % Subtle blue-gray
  \definecolor{accent}{HTML}{5A6572}  % Slightly lighter than secondarytext
 

  \hypersetup{
    pdftitle={Sveinbjörn Geirsson - CV},
    pdfauthor={Sveinbjörn Geirsson},
    pdfsubject={Software Developer Resume},
    pdfkeywords={resume, CV, software developer, cloud infrastructure, Iceland},
    pdfcreator={XeLaTeX},
    pdfproducer={XeLaTeX with hyperref},
    unicode=true,
    pdfborder={0 0 0},
    breaklinks=true,
    colorlinks=true,
    linkcolor=linkcolor,
    filecolor=linkcolor,
    urlcolor=linkcolor,
    pdfnewwindow=true,
    pdfdisplaydoctitle=true
  }


  % Font configuration
  \setmainfont{InterNerdFontPropo}[
      Path = /usr/share/fonts/inter/,
      UprightFont = InterNerdFontPropo-Regular.otf,
      BoldFont = InterNerdFontPropo-Bold.otf,
      ItalicFont = InterNerdFontPropo-Italic.otf,
      BoldItalicFont = InterNerdFontPropo-BoldItalic.otf
  ]

  \newfontfamily{\headingfont}{Iosevka}[
      UprightFeatures={Font=*},
      BoldFeatures={Font=* Bold},
      ItalicFeatures={Font=* Italic},
      BoldItalicFeatures={Font=* Bold Italic}
  ]
  \newfontfamily{\urlfont}{Iosevka}[
      Scale=0.9,
      UprightFeatures={Font=*},
  ]


  \newcommand{\isFlag}{%
    \includesvg[width=1em]{svg/isflag.svg}%
  }

  \newcommand{\udemy}{%
    \includesvg[height=1em]{svg/udemy.svg}%
  }

  \newcommand{\coursera}{%
    \includesvg[height=1em]{svg/coursera.svg}%
  }
 
  \let\certentry\undefined

  \newcommand{\certentry}[5][]{%  % #1=platform (optional), #2=full URL, #3=course name, #4=slug, #5=year
  \noindent
  \begin{minipage}[t]{0.85\linewidth}
    % Handle the optional platform parameter
    \ifx\\#1\\%
      % No platform specified
    \else
      #1\\[0.1cm]% Platform SVG logo on top
    \fi
    \href{#2}{\textbf{#3}} % Course Name as hyperlink\\[0.1cm]
  \end{minipage}%
  \begin{minipage}[t]{0.15\linewidth}
    \raggedleft {\small\textcolor{secondarytext}{#5}}% Year aligned right
  \end{minipage}
  \par\noindent\hfill{\small\href{https://svnbjrn.is/#4}{\textcolor{secondarytext}{\ttfamily svnbjrn.is/#4}}}% Hyperlink for slug
  \vspace{0.4cm}
}


\newcommand{\projectentry}[5]{%  % #1=name, #2=year, #3=subtitle, #4=description, #5=links
    \item 
    \textbf{#1} \hfill \textit{#2}\\
    \textit{#3}\\[0.1cm]
    \begin{minipage}[t]{\linewidth}
        #4
        \vspace{0.2cm}
        #5
    \end{minipage}
    \vspace{0.3cm}
}
 

\newcommand{\projectlink}[3]{%  % #1=full url, #2=icon+text, #3=short url
  \noindent%
  \hspace*{0.35\linewidth}% Reduced indent to pull links left
  \begin{minipage}[t]{0.65\linewidth}% Increased width for long text
    \raggedright%
    \href{#1}{#2}\\[-0.1cm]%
    \hfill{\small\textcolor{secondarytext}{\ttfamily #3}}\\[-0.1cm]
    {\color{secondarytext}\hrulefill}% Separator line
  \end{minipage}%
  \vspace{0.2cm}%
}


  % Create a simpler link command without QR code for reference links
  \newcommand{\reflink}[2]{%  % #1=url, #2=display text
      \href{#1}{#2}\hspace{1em}%
  }

  % Adjust font sizes
  \newcommand{\workentry}[4]{%
    \noindent
    {\headingfont\large\textbf{#1}} \hfill {\small\textit{#2}}\\[0.1cm]     
    {\small\textit{#3}}\\[0.1cm]
    #4
    \vspace{0.3cm}
  }

  \newcommand{\skillgroup}[2]{%
    \textbf{#1}\\
    #2\\[0.3cm]  % Increased from 0.1cm and removed negative spacing
  }

  % Update commands to use Iosevka for headings
  \newcommand{\cvheading}[1]{%
    {\headingfont\section*{#1}}
    \vspace{-0.2cm}
    \hrule
    \vspace{0.2cm}
  }

  % Use Iosevka for the name at the top
  \newcommand{\nameheading}{%
    {\headingfont\LARGE\textbf{Sveinbjörn Geirsson}}
  }

  % Tighter lists
  \setlist{nosep,leftmargin=*}

  % Tighter section spacing
  \titlespacing{\section}{0pt}{*2}{*1}


  \newcommand{\qrlink}[2]{%
    \begin{minipage}[t]{0.7\linewidth}
      \raggedleft
      {\color{linkcolor}\href{#1}{#2}}%
    \end{minipage}%
    \begin{minipage}[t]{0.3\linewidth}
      \raggedleft
      \raisebox{-0.2\height}{%
        \qrcode[height=1.1cm,version=4,level=Q]{#1}%
      }%
    \end{minipage}%
  }




  \makeatletter
  \newcommand{\shorturl}[4][]{%  % #1=icon (optional), #2=full url, #3=short url, #4=display text
      \begin{minipage}[t]{0.85\linewidth}
          \raggedright
          \ifx\\#1\\%
              % No icon
          \else
              {\color{linkcolor}#1}\ %
          \fi
          {\color{linkcolor}\urlfont\href{#2}{#4}}%
      \end{minipage}%
      \begin{minipage}[t]{0.15\linewidth}
          \raggedright
          \raisebox{-0.2\height}{%
              % Use the shortened URL in the QR code, with reduced error correction
              \qrcode[height=0.7cm,version=1,level=L]{#3}%
          }%
      \end{minipage}\\[0.2cm]% Add line break after each shorturl
  }
  \makeatother
  \newcommand{\cvheader}{%
  \vspace*{-0.5cm}  % Negative vertical space to move everything up
  \begin{minipage}[t]{0.45\textwidth}
    \vspace*{-3.5cm}  % Negative vertical space to move everything up
    {\LARGE\textbf{Sveinbjörn Geirsson}}\\[0.3cm]
    \tr{Tölvunarfræðingur}{Software Developer}\\[0.4cm]
    {\color{linkcolor}\href{mailto:sveinbjorn@sveinbjorn.dev}{\faEnvelope\ sveinbjorn@sveinbjorn.dev}}\\[0.2cm]
    {\color{linkcolor}\href{tel:+3548492544}{\faPhone\ (+354) 849 2544}}\\[0.2cm]
    \faMapMarker\ \tr{Vífilsgata 15, 105 Reykjavík}{Vifilsgata 15, 105 Reykjavik, Iceland}\\[0.2cm]
    \faBirthdayCake\ \tr{Apríl 1986}{April 1986}
  \end{minipage}%
  \begin{minipage}[t]{0.55\textwidth}
    \raggedleft
    \begin{tabular}{p{25em}}
      \raggedright\color{linkcolor}\href{https://sveinbjorn.dev}{\faGlobe\ sveinbjorn.dev} \\
      \raggedleft{\small\href{https://svnbjrn.is/cv}{\textcolor{secondarytext}{\ttfamily svnbjrn.is/cv}}} \\[-0.1cm]
      \color{secondarytext}\hrulefill \\[0.2cm]
      \raggedright\color{linkcolor}\href{https://xn--sveinbjrn-67a.is}{\isFlag\ sveinbjörn.is} \\
      \raggedleft{\small\href{https://svnbjrn.is/is}{\textcolor{secondarytext}{\ttfamily svnbjrn.is/is}}} \\[-0.1cm]
      \color{secondarytext}\hrulefill \\[0.2cm]
      \raggedright\color{linkcolor}\href{https://github.com/Raudbjorn}{\faGithub\ Raudbjorn} \\
      \raggedleft{\small\href{https://svnbjrn.is/gh}{\textcolor{secondarytext}{\ttfamily svnbjrn.is/gh}}} \\[-0.1cm]
      \color{secondarytext}\hrulefill \\[0.2cm]
      \raggedright\color{linkcolor}\href{https://linkedin.com/in/sveinbjornG}{\faLinkedin\ sveinbjornG} \\
      \raggedleft{\small\href{https://svnbjrn.is/in}{\textcolor{secondarytext}{\ttfamily svnbjrn.is/in}}} \\[-0.1cm]
      \color{secondarytext}\hrulefill \\[0.2cm]
      \raggedright\color{linkcolor}\href{https://keys.openpgp.org/vks/v1/by-fingerprint/DA1F35544B1F76D1CCE1016A53EB6860D443E32D}{\faKey\ PGP} \\
      \raggedleft{\small\href{https://svnbjrn.is/pgp}{\textcolor{secondarytext}{\ttfamily svnbjrn.is/pgp}}} \\[-0.1cm]
      \color{secondarytext}\hrulefill
    \end{tabular}
  \end{minipage}
  \vspace{-0.2cm}
}


  \begin{document}

  \cvheader

  % Me section
  \cvheading{\sectionMe}
  Hugbúnaðarsérfræðingur með yfir áratugs reynslu í greiðslukortaiðnaðinum, þar sem áhersla hefur verið lögð á áreiðanleika, öryggi og hraða. Drífandi og lausnamiðaður, með áhuga á stöðugri framþróun, sjálfvirkni innviða og þróun vandaðs hugbúnaðar sem er traustur og vel hannaður frá grunni.
  \vspace{0.1cm}  % Reduce extra space created by line break


  % Experience section
  \cvheading{\sectionExperience}

  \workentry{Reiknistofa Lífeyrissjóða}{2023 - 2025}
  {\tr{Hugbúnaðarsérfræðingur}{Software Engineer}}{
    \vspace{-0.4cm}
    \begin{itemize}
      \item \tr{Þróaði og viðhélt hugbúnaðarkerfi sem þjónustar lífeyrissjóði við allt frá húsnæðislánum, verðbréfum, og samskiptum vegna þeirra til eftirlitsaðila, til ráðstöfun á réttindum sjóðsfélaga vegna örorku eða aldurs.}
      {Developed and maintained software systems serving pension funds, handling everything from mortgages, securities, and their regulatory communications, to managing fund members' disability and retirement benefits.}
      
      \item \tr{Annaðist uppsetningu og viðhaldi á X-Road öryggisþjóni; öruggu samskiptalagi sem forritunarskil við stjórnvöld.}
      {Managed the setup and maintenance of X-Road security server; a secure communication layer providing programming interfaces for government services.}
      
      \item \tr{Stór hluti af skyldum var skýrslugjöf til Seðlabanka Íslands; skuldbindingaskrá sem m.a. innheldur stöðu allra lánaveitinga til fasteignakaupa.}
      {A major part of responsibilities involved reporting to the Central Bank of Iceland; including obligations registry containing status of all mortgage lending.}
    \end{itemize}
  }

  \workentry{Rapyd Financial Services, áður Korta}{2013 - 2022}{Hugbúnaðarsérfræðingur}{
    \begin{description}[
      leftmargin=1em,
      labelwidth=0pt,
      itemindent=0pt,
      listparindent=0pt,
      parsep=0.1cm,    % Space between paragraphs within an item
      itemsep=0.1cm    % Space between items
      ]
      \vspace{-0.4cm}  % Reduce extra space created by line break
      \item[\textbf{Uppgjör}]\mbox{}\\ % Force line break after heading
            \vspace{-0.4cm}  % Reduce extra space created by line break
            \begin{itemize}[leftmargin=1em, topsep=0pt]
              \item Útreikningur á kostnaði, veltutryggingu, sölu, endurgreiðslum, o.fl., eftir dagleg boðskipti við kortafélögin, til uppgjörs við söluaðila.
              \item Útbúa, flokka, skipta niður, eða halda eftir greiðslum til söluaðila, bæði innlendum, í gegnum sambankaþjónustur(IOBS/Stórgreiðslukerfi SÍ), og erlendum(SEPA/SWIFT), byggðum á uppgjörsútreikningum og öðrum forsendum s.s. reglum um uppgjörstíðni.
              \item Uppfæra stöðu söluaðila í vildarkerfum.
            \end{itemize}
      \item[\textbf{Fjármál og bókhald}]\mbox{}\\
            \vspace{-0.4cm}
            \begin{itemize}[leftmargin=1em, topsep=0pt]
              \item Samræming Visa/Mastercard afstemmingargagna, og vörpun þeirra á bókhaldslykla, til innlesturs í bókhaldskerfi félagsins.
              \item Ákvarða kostnaðar- og virðisdagsetningar og gjaldmiðil fyrir gjöld á söluaðila, veltutrygginu, sölu og endurgreiðslur, á grundvelli niðurstöðu gjaldaútreiknings, fyrir bókhald félagsins.
              \item Tryggja að kostnaður vegna endurkrafna, svika eða deilumála við korthafa sé færður inn í bókhald félagsins.
            \end{itemize}
      \item[\textbf{Gagnaveiting og endurkröfur}]\mbox{}\\
            \vspace{-0.4cm}
            \begin{itemize}[leftmargin=1em, topsep=0pt]
              \item Miðla stöðu og viðskiptamagni söluaðila til samstarfsaðila vegna greiðsluþjónustu sem veitt var fyrir þeirra hönd.
              \item Innlestur og samræming á gögnum vegna kortasvika eða ágreinings á milli korthafa og söluaðila, og þeirra endurkrafna sem orðið hafa til í kjölfarið.
              \item Þróun og viðhald á stoðgagnaveitingu frá gagnagrunni(DB2) fyrir framendakerfi, sem annars myndi krefjast verulegrar fyrirhafnar eða lénsþekkingar annarra þróunarteyma.
            \end{itemize}
      
      \item[\textbf{Stoðþjónustur}]\mbox{}\\
            \vspace{-0.4cm}
            \begin{itemize}[leftmargin=1em, topsep=0pt]
              \item Innlestur á gengi frá Visa/Mastercard/Seðlabanka Íslands.
              \item Innlestur á gögnum v. endurkrafna/kortasvika/milligjalda.
              \item Ákvarða landfræðileg mörk færslu(t.d. innlend/EES/EU/alþjóðleg osfrv.).
              \item Flokkun færslu eftir vöruframboði kortafélaga (t.d. neytenda-/viðskiptakort, debet-/kreditkort osfrv.).
              \item Samskipti við gagnakskil flugfélaga vegna sérreglna um uppgjörsdag.
            \end{itemize}
    \end{description}
  }
  
  % Skills section
  \cvheading{\sectionSkills}

  % Define column spacing and format
  \newcolumntype{L}[1]{>{\raggedright\arraybackslash}p{#1}}
  
  % Calculate column widths for 3 columns with minimal spacing
  \newlength{\skillcolwidth}
  \setlength{\skillcolwidth}{0.31\textwidth}
  
  % Skills table with three columns
  \begin{tabular}{L{\skillcolwidth}|L{\skillcolwidth}|L{\skillcolwidth}}
  \textbf{Kjarnaþróun} & 
  \textbf{Gagnagrunnskerfi} & 
  \textbf{Netþjónustur} \\
  \begin{itemize}[leftmargin=*]
    \item Java
    \item Python
    \item Scala
    \item JavaScript
    \item TypeScript
    \item C\#
    \item PHP
  \end{itemize} & 
  \begin{itemize}[leftmargin=*]
    \item IBM DB2 (iSeries/LUW)
    \item IBM Informix
    \item MariaDB
    \item MySQL
    \item PostgreSQL
    \item SQLite
    \item Redis
    \item SQL-93+OLAP
    \item SparkSQL
    \item XML+XPath/JSON+GraphQL
    \item Arrow
    \item Parquet
  \end{itemize} & 
  \begin{itemize}[leftmargin=*]
    \item Cloudflare
    \item Route 53
    \item Apache / Nginx / Traefik
    \item X-Road
    \item WebRTC+TURN/STUN
  \end{itemize} \\
  \hline
  
  \textbf{Ský \& innviðir} & 
  \textbf{API \& Árvekni} & 
  \textbf{Öryggi \& Auðkenni} \\
  \begin{itemize}[leftmargin=*]
    \item GCP / AWS / Oracle Cloud
    \item containerd / Docker / Podman
    \item k3s
    \item Ansible
    \item Terraform
    \item KVM
  \end{itemize} & 
  \begin{itemize}[leftmargin=*]
    \item gRPC
    \item OpenAPI/Swagger
    \item Datadog
    \item Prometheus
    \item Sentry
    \item Grafana / Loki
  \end{itemize} & 
  \begin{itemize}[leftmargin=*]
    \item OAuth 2.0
    \item OpenID Connect
    \item Certbot+ACME/DNS01
    \item MX/DKIM/SPF
    \item WireGuard / Tailscale / Shadowsocks
   \end{itemize} \\
  \hline
  
  \textbf{Verkefnastýring \& Skjölun} & & \\
  \begin{itemize}[leftmargin=*]
    \item Jira / Confluence / Bitbucket
    \item Kanban / Agile / Scrum
  \end{itemize} & & \\
  \end{tabular}
  
  \cvheading{\sectionProjects}
  \begin{itemize}[leftmargin=2em]
    \projectentry{missing.cat}{2021}
    {Kötturinn minn týndist, og svo ótrúlega vildi til að þetta lén var laust}
     {Var upphaflega útfært mjög rösklega, og því frekar gróf lausn, en innihélt upphaflega:
      \begin{itemize}
        \item \href{https://ghost.org/}{Ghost CMS} sem framenda.
        \item \href{https://developers.google.com/gmail/api}{Google Workspace API}{fyrir tölvupóstþjónustu.}
        \item Innbyggðan \href{https://developers.facebook.com/docs/messenger-platform/}{Facebook Messenger}, sem tengdist okkur í gegnum "Causes"-síðu.
      \end{itemize}
      }
    {\projectlink{https://missing.cat/}{\faLink\ missing.cat}{svnbjrn.is/cat}}


    \projectentry{ipChecker}{2020}
    {Aðgreinir innlenda umferð frá erlendri}
   {PHP útfærsla á nálguninni sem lýst er af Reykjavík Internet Exchange (RIX) til að skera úr um hvort IP-tala er íslensk án þess að reiða sig á gagnagrunn frá þriðja aðila.\\}
  {\projectlink{https://github.com/Raudbjorn/ipChecker}{\faGithub\ Raudbjorn/ipChecker}{svnbjrn.is/ip}}
  {\projectlink{https://www.rix.is/is-as-nets}{\faLink\ rix.is}{svnbjrn.is/rix}}

  

  \projectentry{Knapsack Optimizer Service}{2019}
  {Hæfnispróf}
  {Gámavædd, ósamstillt, REST-leg, skyndiminnug vefþjónusta, sem getur leyst Hnappapokavandamálið - frægt vandamál í tölvunarfræði. Leyst sem hluti af ráðningarferli stórrar alþjóðlegrar samsteypu.\\}
  {\projectlink{https://github.com/Raudbjorn/knapsack-optimizer-service}{\faGithub\ Raudbjorn/knapsack-optimizer-service}{svnbjrn.is/sack}}
 
  \newpage

 \projectentry{Leiguvaktin.is}{2014}
   {Staða framboðs á leigumarkaði Reykjavíkur}
   {Tilraunagæluverkefni sem miðar að því að koma samantekt á lifandi gögnum um íbúðir til leigu á Íslandi frá mörgum auglýsendum á einn stað.}  
   {\projectlink{https://github.com/Raudbjorn/leiguvaktin}{\faGithub\ Raudbjorn/leiguvaktin}{svnbjrn.is/rent-src}}
  {\projectlink{https://web.archive.org/web/20160111052408/http://leiguvaktin.is/}{\faLink\ leiguvaktin.is}{svnbjrn.is/rent}}


 \end{itemize}

  \cvheading{\sectionCertifications}
  \begin{multicols}{2}
    % Left column
    \certentry[\udemy]{https://www.udemy.com/certificate/UC-a18343db-3291-4e45-849b-68a584aaac40}
    {DevOps Ansible Automation}{nsbl}{2023}
  
    \certentry[\udemy]{https://www.udemy.com/certificate/UC-P60ZSDBZ}
    {The Rust Programming Language}{rst}{2018}
  
    \certentry[\coursera]{https://www.coursera.org/account/accomplishments/verify/6ZECRXNYKFSA}
    {Functional Programming in Scala}{scl}{2016}
  
    \columnbreak
  
    % Right column
    \certentry[\coursera]{https://www.coursera.org/account/accomplishments/verify/VB7GCM8TMN}
    {Usable Security}{usc}{2015}
  
    \certentry[\coursera]{https://www.coursera.org/account/accomplishments/verify/MYTYH9LQ7V}
    {Cryptography}{cry}{2015}
  
    \certentry[\coursera]{https://www.coursera.org/account/accomplishments/verify/RPFNLVGPF3}
    {Software Security}{ssc}{2014}
  \end{multicols}

  \cvheading{\sectionAwards}

  \begin{itemize}[leftmargin=2em, itemsep=0.2em]

    \projectentry{Rapyd Internal Hackathon}{2021}
    {Sigurvegari}
    {Teymið aðlagaði Rapyd Wallet að WhatsApp Business API; úrlausnin var gámuð, fullvirk, prótótípa, þar sem viðskiptavinur getur:
      \begin{itemize}[leftmargin=2em, topsep=0.2em]
        \item Borgað fyrir pöntun á veitingastað innan úr WhatsApp með greiðsluhlekk eða geymdum skilríkjum.              
        \item Notað WhatsApp til að biðja tengilið um að greiða fyrir pöntunina; við samþykki er greiðslu ráðstafað úr veski tengiliðsins.              
        \item Óskað eftir að tengiliður greiði, og að fengnu samþykki innan WhatsApp, myndi leyfa notkun á veskinu sínu fyrir viðskiptin.
      \end{itemize}
    }
    {\projectlink{https://community.rapyd.net/t/whatsapp-checkout/1667}{\faNewspaper\ rapyd.net}{svnbjrn.is/rapyd}}
    {\projectlink{https://github.com/Rapyd-Samples/whatsapp-checkout}{\faGithub\ Rapyd-Samples/whatsapp-checkout}{svnbjrn.is/wacheckout}}

  \end{itemize}


% Explicitly select the language here if needed
 
\cvheading{\sectionReferences}
\begin{multicols}{2}
\raggedright
% First reference
\textbf{Tómas Árni Jónsson}\\[0.2cm]
Hugbúnaðarsérfræðingur @ Deloitte\\[0.4cm]
\textit{Samstarf við aðlögun kjarnakerfa RL að gagnaskilakerfi Seðlabanka Íslands, 2023--2024}\\[0.4cm]
\href{tel:+3548926309}{{\color{accent}\faPhone\hspace{0.3em}+354 892 6309}}\\[0.4cm]
\href{mailto:tomasarni@gmail.com}{{\color{accent}\faEnvelope\hspace{0.3em}tomasarni@gmail.com}}

\columnbreak

% Second reference
\textbf{Þráinn Guðbjörnsson}\\[0.2cm]
Áhættustjóri @  Festa Lífeyrissjóður\\[0.4cm]
\textit{Þróun fyrstu kerfa Korta sem annar og þriðji forritari fyrirtækisins, 2013–2017.}\\[0.4cm]
\href{tel:+3548638308}{{\color{accent}\faPhone\hspace{0.3em}+354 863 8308}}\\[0.4cm]
\href{mailto:thrainn@gmail.com}{{\color{accent}\faEnvelope\hspace{0.3em}thrainn@gmail.com}}
\end{multicols}


\end{document}
